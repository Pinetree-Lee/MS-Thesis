% -*- TeX:UTF-8 -*-
%%
%% KAIST 학위논문양식 LaTeX용 (ver 0.5) 예시
%%
%% @version 0.4
%% @author  채승병 Chae,Seungbyung (mailto:chess@kaist.ac.kr)
%% @date    2004. 11. 12.
%%
%% @requirement
%% teTeX, fpTeX, teTeX 등의 LaTeX2e 배포판
%% + 은광희 님의 HLaTeX 0.991 이상 버젼 또는 홍석호 님의 HPACK 1.0
%% : 설치에 대한 자세한 정보는 http://www.ktug.or.kr을 참조바랍니다.
%%
%% @note
%% 기존에 널리 쓰여오던 차재춘 님의 학위논문양식 클래스 파일의 형식을
%% 따르지 않고 전면적으로 다시 작성하였습니다. 논문 정보 입력부분에서
%% 과거 양식과 다른 부분이 많으니 아래 예시에 맞춰 바꿔주십시오.
%%
%%
%% @acknowledgement
%% 본 예시 논문은 물리학과 박사과정 김용현 님의 호의로 제공되었습니다.
%%
%% -------------------------------------------------------------------
%% @information
%% 이 예제 파일은 hangul-ucs를 사용합니다. UTF-8 입력 인코딩으로
%% 작성되었습니다. hlatex의 hfont는 이용하지 않습니다. --2006/02/11
%% 본 템플릿은 전산학부 김민혁 교수에의해서 버그 수정되었습니다. -- 2016/11/25

% @class kaist.cls
% @options [default: doctor, korean, final]
% - doctor: 박사과정 | master : 석사과정
% - korean: 한글논문 | english: 영문논문
% - final : 최종판   | draft  : 시험판
% - pdfdoc : 선택하지 않으면 북마크와 colorlink를 만들지 않습니다.


\documentclass[master,english,draft,pdfdoc]{kaist-ucs}


% If you want make pdf document (include bookmark, colorlink)
%\documentclass[doctor,english,final,pdfdoc]{kaist-ucs}

% kaist.cls 에서는 기본으로 dhucs, ifpdf, graphicx 패키지가 로드됩니다.
% 추가로 필요한 패키지가 있다면 주석을 풀고 적어넣으십시오,
%\usepackage{...}


% @command title 논문 제목(title of thesis)
% @options [default: (none)]
% - korean: 한글제목(korean title) | english: 영문제목(english title)
\title[korean] {액상 냉각 매니폴드 마이크로채널 히트싱크의 열성능 예측}
\title[english]{Thermal performance prediction of liquid-cooled manifold microchannel (MMC) heat sinks}

% @note 표지에 출력되는 제목을 강제로 줄바꿈하려면 \linebreak 을 삽입.
%       \\ 나 \newline 등을 사용하면 안됩니다. (아래는 예시)
%
%\title[korean]{탄소 나노튜브의 물리적 특성에 대한\linebreak 이론 연구}
%\title[english]{Theoretical study on physical properties of\linebreak
%                carbon nanotubes}
%
% If you want to begin a new line in cover, use \linebreak .
% See examples above.
%


% @command author 저자 이름
% @param   family_name, given_name 성, 이름을 구분해서 입력
% @options [default: (none)]
% - korean: 한글이름 | chinese: 한문이름 | english: 영문이름
% 한문 이름이 없다면 빈 칸으로 두셔도 됩니다.
%
%
% If you are a foreigner , write your name in korean or your korean name.
% If you can't write native character, you can make the chinese blank empty 
% Write as follow
% \author[korean]{family name in korean}{given name in korean}
% \author[chinese]{family name in your native language}{given name in your native language}
% \author[english]{family name in english}{given name in english}
%
\author[korean] {이}{한 솔}
\author[korean2] {이}{한솔}    %이름을 붙여 써 주시기 바랍니다.
\author[chinese]{李}{한 솔}
\author[english]{Lee}{Hansol}

% @command advisor 지도교수 이름 (복수가능)
% @usage   \advisor[options]{...한글이름...}{...영문이름...}{signed|nosign}
% @options [default: major]
% - major: 주 지도교수  | coopr: 공동 지도교수
\advisor[major]{김 성 진}{Sung Jin Kim}{signed}
\advisor[major2]{김성진}{Sung Jin Kim}{signed}    %한글 성과 한글 이름을 모두 붙여 써 주시기 바랍니다.
\advisorinfo{Professor of Mechanical Engineering} %제출승인서에 들어가는 교수님 정보, advisor's information 
%\advisor[coopr]{홍 길 동}{Gil-Dong Hong}{nosign}
%\advisor[coopr2]{홍길동}{Gil-Dong Hong}{nosign}    %한글 성과 한글 이름을 모두 붙여 써 주시기 바랍니다.
%
% 지도교수 한글이름은 입력하지 않아도 됩니다.
% You may not input advisor's korean name
% like this \advisor[major]{}{Chang, Kee Joo}{signed}
%


% @command department {학과이름}{학위종류} - 아래 규칙에 따라 코드를 입력
% @command department {department code}{degree field}
%
% department code
% 2. 석박사학위논문 작성 및 제출요령 4쪽 ~ 5쪽 참고
% 또는 kaist-ucs.cls 의 % @command department 참고

% science: 이학 | engineering: 공학 | business : 경영학
% 박사논문의 경우는 학위종류를 입력하지 않아도 됩니다.
% If you write Ph.D. dissertation, you cannot input degree field.
% The third parameter : a | b | c
% a: 소속된 학과만 쓰는 옵션 (학과에만 소속되어 있는 경우에는 무조건 a를 선택해야 함)
% b: 학과 아래의, 프로그램이나 학제전공에 소속되어 있을 경우에 학과와 프로그램을 함께 쓰는 옵션
% c: 학과 아래의, 프로그램이나 학제전공에 소속되어 있을 경우에 학과를 쓰지 않고 프로그램이나 학제전공의 이름만 쓰는 옵션 
% 
% a: it represents only the name of department. (if you aren't in the program under the department, must choose a)
% b: it represents the names of department and the program that is under the department (consider this when you are in the program not only department)
% c: it represents only the name of program that is under the department (consider this when you are in the program not only department)
\department{ME}{engineering}{a}

% @command referee 심사위원 (석사과정 3인, 박사과정 5인)
\referee[1]{김 성 진}
\referee[2]{안 진 현}
\referee[3]{정 태 성}

% \referee[5] {Barack Obama}
% Of course english name is available

% @command approvaldate 지도교수논문승인일
% @param   year,month,day 연,월,일 순으로 입력
\approvaldate{2023}{12}{10}

% @command refereedate 심사위원논문심사일
% @param   year,month,day 연,월,일 순으로 입력
\refereedate{2023}{12}{10}

% @command gradyear 졸업년도
\gradyear{2024}

% 본문 시작
\begin{document}

    % 앞표지, 속표지, 학위논문 제출승인서, 학위논문 심사완료 검인서는
    % 클래스 옵션을 final로 지정해주면 자동으로 생성되며,
    % 반대로 옵션을 draft로 지정해주면 생성되지 않습니다.

    % 논문 서지, 초록, 핵심 낱말, 영문 초록, 영어 핵심 낱말 (Information of thesis, abstract in korean, keywords in korean, abstract in english, keywords in english)
   \thesisinfo
   %% Letters of abstract in korean must be less than 500 and words of abstract in english must be less than 300.
   %% Number of keywords must be less than 6.
   %% Don't write english letters in the abstract in korean.
    \begin{summary}      


    \end{summary}
   
    \begin{Korkeyword}

    \end{Korkeyword}


    \begin{abstract}

    \end{abstract} 
     
    \begin{Engkeyword}

    \end{Engkeyword}
   

    \addtocounter{pagemarker}{1}                 % 백색별지분을 고려
    \newpage 
  


    % 목차 (Table of Contents) 생성
    \tableofcontents

    % 표목차 (List of Tables) 생성
    \listoftables

    % 그림목차 (List of Figures) 생성
    \listoffigures

    % 위의 세 종류의 목차는 한꺼번에 다음 명령으로 생성할 수도 있습니다.
    %\makecontents

%% 이하의 본문은 LaTeX 표준 클래스 report 양식에 준하여 작성하시면 됩니다.
%% 하지만 part는 사용하지 못하도록 제거하였으므로, chapter가 문서 내의
%% 최상위 분류 단위가 됩니다.
%% You cannot use 'part'

\chapter{Introduction}
\section{Background}
\noindent
In response to the global push to reduce carbon emissions, electrification has become a growing trend. However, the increasing use of electronic devices, including solar cells, radio frequency (RF) power amplifiers, and power converters, has led to a severe problem of thermal management failure in recent years. These devices are based on wide bandgap (WBG) semiconductors such as GaN and SiC, which are well-known for their superior physical characteristics of high electrical breakdown voltage and high electron charge density. However, the full potential of these semiconductors is limited by thermal constraints. Therefore, adopting viable cooling solutions for electronic devices is crucial to alleviate these thermal constraints and ensure optimal performance.

Among possible cooling solutions, 3-D manifold microchannel (MMC) heat sinks have drawn attention as a potential solution to address thermal failures in electronic devices. 3-D MMC heat sinks consist of a fluid-distributing manifold layer and a microchannel heat sink layer where the heat transfer between working fluid and solid interfaces mainly occurs. Since the length of the fluid path across the heat sink layer becomes reduced, the heat transfer performance of MMC heat sinks is significantly higher than conventional 2-D parallel microchannel heat sinks at the given pressure drop. 

Although 3-D MMC heat sinks have superior thermal performance, it is still hard to embed this cooling solution in electronic devices since predicting the thermo-hydraulic performance of 3-D MMC heat sinks has been challenging. Previous models have been limited in accuracy, as they assumed uniform mass flow rate distribution along each microchannel and considered the heat transfer characteristic of developing flow only. To accurately predict the thermo-hydraulic performance of MMC heat sinks, first, the distribution of mass flow rates across each microchannel and pressure drop should be considered. Second, heat transfer characteristics of developing flow and jet impingement, which are attributed to the U-shaped flow along each microchannel, should be taken into account in the model.

\section{Literature review}
The concept of 3-D manifold microchannel (MMC) heat sinks was originally proposed by Harpole and Eninger \cite{Harpole1991}. They conducted a parametric study on the thermal performance of 3-D MMC heat sinks by using the numerical model. The result indicates that the number of manifold channels per unit width, microchannel width, inlet/outlet width affect the thermal performance of 3-D MMC heat sinks. Based on the optimized parameters, they contended that the effective heat transfer coefficient of \(10^6\) W/m\(^2\)-K can be achievable while maintaining the pressure drop below 2 bar. Although the pressure drop was expected to be very high, the results shed light on the possibility of cooling high heat flux electronics above 1 kW/cm\(^2\) using 3-D MMC heat sinks. 

Copeland \textit{et al.} \cite{Copeland1995} analytically and experimentally investigated the thermal performance of 3-D manifold microchannel heat sinks. Based on the analytical model, they anticipated the thermal performance of 3-D MMC heat sinks; however, the experimental results showed lower thermal resistance compared to the prediction by the model. Also, unlike the prediction, the thermal resistance of the 3-D MMC heat sink increases with the increase in the channel height at the fixed volume flow rate.


\section{Objectives and major contributions}
The goal of the research 

\section{Organization of the dissertation}


\chapter{Thermo-hydraulic modeling of manifold microchannel heat sinks}

\section{Quantum Communication}


\subsection{Examples}


\section{Applications of Quantum Communication}



\chapter{Experimental validation}

\section{Proposed Architecture}


%%
%% 표 삽입 예시
%% Example. how to insert table
%%
\begin{table}[t]
\caption[Enter the caption title here]{Energy stability $E$ (eV) per molecule of all meta-stable
isomer states of C$_{60}$ opening process for forming the (5,5) cap.
In the SW-I and SW-II, both ferromagnetic (Ferro) and paramagnetic (Para)
spin configurations are obtained, whereas only non-magnetic configuration
is obtained in the BF, SW-III, and CAP(5,5).
$M$ is total magnetization $n_{\rm up}$-$n_{\rm down}$ in unit of $\mu_B$, where
$n_{\rm up(down)}$ is the number of up (down) spins.
}
\label{mag-tab1}
\begin{center}
\begin{tabular} {ccccccccccc}
\hline\hline
& & BF &\multicolumn{2}{c}{SW-I}&&\multicolumn{2}{c}{SW-II}&SW-III&CAP&\\
\cline{4-5} \cline{7-8}
&               &   &  Para & Ferro &&   Para &  Ferro &      &      &\\
\hline
& $E$ (eV)      & 0 & 7.796 & 7.832 && 10.418 & 10.408 & 11.5 & 13.2 &\\
& $M$ ($\mu_B$) & 0 &     0 &  1.94 &&      0 &   2.06 &    0 &    0 &\\
\hline\hline
\end{tabular}
\end{center}
\end{table}

%%
%% 그림 삽입 예시
%% Example. how to insert graph
%%
%% Note. 가급적 \includegraphics 명령을 사용하십시오.
%% Recommen : Use \includegraphics to insert graph.
%%


\section{Application}


\chapter{Thermal performance optimization of manifold microchannel heat sinks}

\chapter{Concluding Remark}


%%
%% 참고문헌 시작
%% bibliography
%% It can be changed but should include sufficient information.
\bibliographystyle{elsarticle-num}
\bibliography{refs}



%%
%% 감사의 글 시작
%% Acknowledgement
%%
% @command acknowledgement 감사의글
% @options [1 | 2 | 3 |4 ]
% - 1 : 본문과 감사의 글이 둘 다 한글일 때  | 2 : 본문은 한글인데 감사의 글이 영어일 때 | 3 :  본문과 감사의 글이 둘 다 영어일 때  | 4 : 본문은 영어인데 감사의 글이 % 한글일 때 
%% It is optional.

\acknowledgment[4]
    언제나 저를 바른 길로 이끌어 주시는 송익호 교수님께 큰 고마움을 느낍니다.
    끝으로 오늘의 제가 있을 수 있도록 사랑으로 키워 주신 가족들에게 감사드립니다.
    저의 이 작은 결실이 그분들께 조금이나마 보답이 되기를 바랍니다.

%%
%% 약력 시작
%% Curriculum Vitae
%%
% @command curriculumvitae 이력서
% @options [1 | 2 | 3 |4 ]
% - 1 : 본문과 약력이 둘 다 한글일 때  | 2 : 본문은 한글인데 약력이 영어일 때 | 3 :  본문과 약력이 둘 다 영어일 때  | 4 : 본문은 영어인데 약력이 한글일 때 
%% It is optional and you can change form of this in the class file if you want.
\curriculumvitae[4]

    % @environment personaldata 개인정보
    % @command     name         이름
    %              dateofbirth  생년월일
    %              birthplace   출생지
    %              domicile     본적지
    %              address      주소지
    %              email        E-mail 주소
    % - 위 6개의 기본 필드 중에 이력서에 적고 싶은 정보를 입력
    % input data only you want
    \begin{personaldata}
        \name       {이 한 솔}
        \dateofbirth{1997}{06}{20}
        \birthplace {대한민국}
        \address    {대전광역시}
    \end{personaldata}

    % @environment education 학력
    % @options [default: (none)] - 수학기간을 입력
    \begin{education}
        \item[2008. 3.\ --\ 2010. 2.] 고등학교 (2년 수료)
        \item[2010. 2.\ --\ 2014. 2.] 한국과학기술원 물리학과 (학사)
        \item[2014. 3.\ --\ 2016. 2.] 한국과학기술원 전기및전자공학부 (석사)
    \end{education}

    % @environment career 경력
    % @options [default: (none)] - 해당기간을 입력
    \begin{career}
        \item[2015. 3.\ --\ 2016. 2.] 한국과학기술원 전기및전자공학부 조교
    \end{career}

    % @environment activity 학회활동
    % @options [default: (none)] - 활동내용을 입력
%%    \begin{activity}
%%        \item J. Choi, \textbf{Yong-Hyun Kim}, K.J. Chang, and D. Tomanek,
%%             \textit{Occurrence of itinerant ferromagnetism in C/BN superlattice
%%             nanotubes}, 5th Asian Workshop on First-Principles Electronic
%%             Structure Calculations, Seoul (Korea), October., 2002.
%%    \end{activity}

    % @environment publication 연구업적
    % @options [default: (none)] - 출판내용을 입력
    \begin{publication}
        %\item \textbf{Yong-Hyun Kim}, J. Choi, K.J. Chang, and D. Tomanek,
        %     \textit{Magnetic instability in partly opened C$_{60}$ isomers},
        %     in preparation.
        \item H.-K. Min, Y. Hou, {\bf S. Park}, and I. Song,
``A computationally efficient scheme for feature extraction with kernel discriminant analysis,"
\textit{Patt. Recogn.}, vol.~50, no.~2, pp.~45-55, Feb. 2016 (to be published).
    \end{publication}

  \label{paperlastpagelabel}     % <-- 추가 부분: 마지막 페이지 위치 지정	
%% 본문 끝
\end{document}
